\documentclass[journal]{IEEEtran}

\usepackage[T1]{fontenc}
\usepackage[utf8]{inputenc}
\usepackage{lmodern}

\usepackage{ae}
% \usepackage[brazilian]{babel}
\usepackage{hyphenat}
\usepackage{fancyhdr}
\usepackage{float}
\usepackage{cite}
\usepackage[pdftex]{hyperref}
\usepackage[pdftex]{color,graphicx}

\usepackage[cmex10]{amsmath}

\usepackage{array}

\usepackage{mdwmath}
\usepackage{mdwtab}
\usepackage{url}

\hyphenation{op-tical net-works semi-conduc-tor con-si-de-ram}

\begin{document}
\title{Um Estudo da Instabilidade de Saffman-Taylor com Fluido Magnético e, ou Anisotrópico}

% author names and IEEE memberships
% note positions of commas and nonbreaking spaces ( ~ ) LaTeX will not break
% a structure at a ~ so this keeps an author's name from being broken across
% two lines.
% use \thanks{} to gain access to the first footnote area
% a separate \thanks must be used for each paragraph as LaTeX2e's \thanks
% was not built to handle multiple paragraphs

\author{Ataias~Pereira~Reis, Yuri~Dumaresq~Sobral, Francisco~Ricardo~da~Cunha\\Universidade de Brasília\\Departamento de Matemática\\Brasília, Brasil\\ataiasreis@gmail.com}

% The paper headers
\markboth{ProIC, Julho~2013}%
{Shell \MakeLowercase{\textit{et al.}}: Bare Demo of IEEEtran.cls for Journals}

\maketitle


\begin{abstract}
%\boldmath
Criação de algoritmos para resolver equações diferenciais parciais. Uso de diferenças finitas para resolver as equações de Laplace e Poisson, usando método ímplicito e explícito, usando bibliotecas de álgebra linear de código aberto. Análise da diferença de tempo de resolução entre os métodos implícito e explícito, e no caso do método explícito. Uso do método iterativo e de resolução direta de sistema linear. 
\end{abstract}
% IEEEtran.cls defaults to using nonbold math in the Abstract.
% This preserves the distinction between vectors and scalars. However,
% if the journal you are submitting to favors bold math in the abstract,
% then you can use LaTeX's standard command \boldmath at the very start
% of the abstract to achieve this. Many IEEE journals frown on math
% in the abstract anyway.

% Note that keywords are not normally used for peerreview papers.
\begin{IEEEkeywords}
Ferrofluidos, Navier~Stokes, EDP, Poisson, Laplace, Diferenças~finitas, Eigen
\end{IEEEkeywords}

% For peer review papers, you can put extra information on the cover
% page as needed:
% \ifCLASSOPTIONpeerreview
% \begin{center} \bfseries EDICS Category: 3-BBND \end{center}
% \fi
%
% For peerreview papers, this IEEEtran command inserts a page break and
% creates the second title. It will be ignored for other modes.
\IEEEpeerreviewmaketitle

\section{Introdução}
A instabilidade de Saffman-Taylor, o endedamento, ou \textit{fingering}, ocorre na superfície de contato entre dois fluidos. Ocorre quando um fluido menos viscoso é injetado para deslocar um outro mais viscoso (na situação inversa, do fluido mais viscoso usado para movimentar o outro, a interface é estavel, não ocorre o endedamento). Também pode ocorrer movida pela gravidade, ao invés de injeção de um fluido em outro. Neste caso, se a interface separando os fluidos de diferentes densidades está direcionada na horizontal, e o mais pesado em cima do outro. Este tipo de problema ocorre com a injeção de água em tubos de petrolíferas marítimas, para fazer o óleo subir. Este tipo de efeito é negativo na extração do petróleo, pois ocorrem bolhas de um líquido no outro que causam problemas para se obter somente o óleo.

Para o estudo dessa instabilidade, que não é nada trivial, precisa-se de ferramentas númericas. A equação de Navier Stokes deve ser discretizada e então resolvida numericamente. Ela é uma equação diferencial parcial altamente não-linear e de difícil resolução. No caso da instabilidade de Saffman-Taylor, a fronteira se movimenta, que é a interface entre os dois fluidos, então também precisa-se de um método que lide bem com isso. 

Tais ferramentas numéricas e decisão de métodos/algoritmos a se utilizar são a primeira etapa neste projeto. A apresentação no resto deste relatório mostrará até onde se alcançou na codificação dos algoritmos para resolução do problema proposto, que inicia-se com o estudo de equações diferenciais parciais e de diferenças finitas, o método utilizado para discretização, e o uso em partes da equações de Navier Stokes até chegar em sua totalidade.
\section{Metodologia}
A metologia aqui proposta tenta-se seguir uma ordem mais ou menos cronológica das tarefas e do que estava sendo utilizado ou não.
\begin{enumerate}
  \item Equação de Laplace
  \item Ferramenta
  \item Método Explícito
  \item Método Implícito
  \item git e python
  \item Navier Stokes
\end{enumerate}
\subsection{Equação de Laplace}
\subsubsection{Forma contínua}
Primeiramente, teve-se a familiarização do aluno com a resolução de equações diferenciais parciais, realizando um curso na UnB, onde foram vistas principalmente a equação de Laplace, Poisson, Calor e Onda. O método de resolução aprendido foi por separação de variáveis, o método de Fourier. Após isso, o primeiro problema proposto foi resolver uma destas equações numericamente, a escolhida foi a equação de Laplace em duas dimensões, que segue abaixo:
\begin{equation}
\nabla^2 u=\frac{\partial^2 u}{\partial x^2}+\frac{\partial^2 u}{\partial y^2}=0\label{laplace}
\end{equation}

A primeira etapa na resolução desta equação de maneira numérica é discretizá-la, e isto quer dizer limitar o domínio, escolher um número de pontos em cada dimensão, e representar a equação como operações mais simples que o computador possa entender, diferentes da derivada. O escolhido foi utilizar as diferenças finitas.
\subsubsection{Forma discreta - explícito}
Para a discretização da equação \ref{laplace}, usa-se o método mencionado antes, que são as diferenças finitas. Este método basicamente faz a seguinte aproximação para a derivada:
\begin{equation}
f'(x)=\lim_{h\rightarrow 0}\frac{f(x+h)-f(x)}{h}=\frac{d}{dx}f(x)\approx \frac{f_i-f_{i-1}}{\Delta x}
\end{equation}
Os valores de $f_i$ são valores discretizados da função $f(x)$, ou seja, $f_i=f(i\Delta x)$. Agora a distância entre dois pontos passa a ser de $\Delta x$, e daí a derivada se calcula como a diferença entre um ponto e outro, dividida pela distância entre eles. Há vários tipos de diferenças: progressivas, regressivas, centrais e podem ter mais pontos, e diferentes ordens de erro podem ser obtidas dependendo dos valores de $\Delta x$ e do número de pontos usados nos cálculos. Para Laplace, utilizou-se diferenças centrais de segunda ordem. Neste caso, o erro é da ordem de $\Delta x^2$.
\begin{equation}
f''(x)\approx \frac{f(x+\Delta x)-2f(x)-f(x-\Delta x)}{\Delta x^2}
\end{equation}
Utilizando essa equação para laplace, chegamos na seguinte discretização, considerando $\Delta x=\Delta y$ e que a variável $x$ varia com pontos indicados pelo indice $i$ e a variável $y$ com o índice $j$:
\begin{equation}
\frac{u_{i+1j}+u_{i-1j}+u_{ij+1}+u_{ij-1}-4u_{ij}}{\Delta x^2}=0
\end{equation}
Reordenando essa equação:
\begin{equation}
  u_{ij}=\frac{u_{i+1j}+u_{i-1j}+u_{ij+1}+u_{ij-1}}{4}
\end{equation}
Esta é a forma explícita, na qual um ponto se obtém a partir de outros pontos, mas como quando se muda o valor de um ponto, ele influencia no cálculo de outros, esta fórmula deve ser aplicada muitas vezes, até se alcançar a convergência do domínio.
\section{Resultados}
\newpage
\subsection{Programas\label{exemplos}}
\newpage

\subsection{Acessar dois Kinects}


\section{Conclusão}

% use section* for acknowledgement
\section*{Agradecimentos}


The authors would like to thank...

\begin{thebibliography}{4}
  
\bibitem{ubuntu} \href{ftp://ftp.fisio.cinvestav.mx/Manuales/linux/Getting\%20Started\%20with\%20Ubuntu\%2010.10.pdf}{Getting Started with Ubuntu 10.10}
\bibitem{openkinect} \href{http://openkinect.org/wiki/Main_Page}{OpenKinect}
\end{thebibliography}

\end{document}
