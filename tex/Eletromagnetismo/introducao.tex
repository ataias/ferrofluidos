\documentclass[introducao.tex]{subfiles}
\begin{document}
\section{Introdução}
\paragraph{} Quando se simula a dinâmica de um fluido magnético numa cavidade, é importante definir a sequência de passos nos quais valores intermediários devem ser calculados para então se resolver a hidrodinâmica. Note que para $t_0$, há condições iniciais que determinam o estado do sistema. Para $t_n > t_0$, tem-se:

\begin{enumerate}
	\item Obter $\phi_n$ a partir de $\mathbf{M}_{n-1}$ com condições de contorno apropriadas
	\item Calcular campo dentro da cavidade por meio de $\mathbf{H} = -\nabla \phi$
	\item Obter $\mathbf{M}_{n}$ a partir de uma relação adequada
	\item Calcular força $\mathbf{f} = \mathrm{Cpm}\,\mathbf{M}\cdot \nabla \mathbf{H}$
	\item Resolver hidrodinâmica
	\item Avançar para $t_{n+1}$ e repetir passos anteriores
\end{enumerate}

\paragraph{} Estas etapas serão analisadas neste documento.

\subsection{Matriz de rotação}
\paragraph{} É útil ter um método de se rotacionar o campo magnético para se testar diferentes configurações do mesmo e analisar simetrias e assimetrias no nosso domínio. Como nosso problema é 2D, tem-se:

\begin{eqnarray}
R(\theta) & =& \left[\begin{array}{cc}
\cos(\theta) & -\sin(\theta)\\
\sin(\theta)  & \cos(\theta)
\end{array}\right]
\end{eqnarray}
\paragraph{} Esta é uma matriz de rotação, logo:
\begin{itemize}
	\item $\det R(\theta) = 1$;
	\item $R^{-1}(\theta) = R^T(\theta)$.
\end{itemize}

\paragraph{} Considere o campo magnético $\mathbf{f}(\mathbf{x})$, no qual $\mathbf{x} = (x,y)$, rotacionado de um ângulo $\theta$. Neste caso, o ponto $\mathbf{x}$ será dado pelo seguinte:

\begin{eqnarray}
\mathbf{f}'(\mathbf{x}) & = & R\cdot \mathbf{f}(R'\cdot \mathbf{x})
\end{eqnarray}

\paragraph{} \textit{Dúvida: se for deslocar o centro do campo magnético, devo fazer isso antes ou depois de rotacionar? Se $\mathbf{f}$ não é linear, fará diferença.}
\end{document}
