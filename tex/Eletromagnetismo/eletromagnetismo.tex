\documentclass[eletromagnetismo.tex]{subfiles}
\begin{document}
\section{Eletromagnetismo}

\paragraph{} Esta é a hora na qual as equações do eletromagnetismo serão introduzidas neste projeto. Até a última seção, a força utilizada era nula. A partir deste momento, iniciar-se-á um estudo matemático do comportamento de um fluido magnético na presença de um campo magnético. Iniciaremos com algoritmos básicos e não tão precisos e após isso iremos melhorar nosso ferramental. 

\subsection{Equações Básicas}

\paragraph{} Primeiramente, temos a equação do campo magnético e de seu divergente. O fato do divergente ser 0 indica que o campo é solenoidal, isto é, todas as linhas de campo magnético que saem, tem de voltar à superfície.

\begin{eqnarray}
\mathbf{B} &=& \mu_0(\mathbf{M}+\mathbf{H})\\
\nabla\cdot \mathbf{B} & = & 0
\end{eqnarray}

\paragraph{} A consequência das equações acima é que 

\begin{eqnarray}
\nabla\cdot\mathbf{H} = -\nabla\cdot \mathbf{M}	\label{mh1}
\end{eqnarray}


No caso de nosso estudo, temos um caso magnetostático, o que implica que:

\begin{eqnarray}
\nabla\times \mathbf{H} & = & \mathbf{0}
\end{eqnarray}

\paragraph{} Lembre-se que $\mathbf{H}$ é o fluxo elétrico. Como este campo é irrotacional, temos que $\mathbf{H}$ é um campo potencial:

\begin{equation}
\mathbf{H} = -\nabla\phi \label{poth}	
\end{equation}

\paragraph{} A partir das Equações \ref{mh1} e \ref{poth}, tem-se:

\begin{align}
\nabla^2\phi &= \nabla\cdot \mathbf{M}\label{eqcampomag}\\ \eqname{Equação do Campo Magnético}
\end{align}

\subsection{Escolha do Campo Aplicado}

\subsubsection{Atualmente}
\paragraph{} Campo aplicado de um fio infinito conforme \cite{Tzirtzilakis2013}: \begin{eqnarray}
\overline{H}_x & = & \frac{\gamma}{2\pi}\frac{\overline{y}-\overline{b}}{(\overline{x}-\overline{a})^2+(\overline{y}-\overline{b})^2},\\
\overline{H}_y & = & -\frac{\gamma}{2\pi}\frac{\overline{x}-\overline{b}}{(\overline{x}-\overline{a})^2+(\overline{y}-\overline{b})^2}.
\end{eqnarray}

\paragraph{} As equações que acabaram de ser apresentadas são dimensionais. Elas são tornadas adimensionais utilizando os termos: comprimento característico, $L$; e escala de campo magnético $H_0$. Para $\overline{H}_x$, tem-se \begin{eqnarray}
H_x\cdot H_0 & = & \frac{\gamma}{2\pi}\frac{y\cdot L-b\cdot L}{(x\cdot L-a\cdot L)^2+(y\cdot L-b\cdot L)^2}
\end{eqnarray} que resulta em \begin{eqnarray}
H_x & = & \frac{1}{2\pi}\frac{\gamma}{L H_0}\frac{y - b}{(x-a)^2+(y-b)^2}
\end{eqnarray}

\subsubsection{Posteriormente}
\paragraph{} Para definir o campo aplicado $\mathbf{H}$, nos basearemos no trabalho de McCaig e Clegg \cite{mccaig}. O livro citado apresenta uma fórmula que condiz com a realidade muito bem. O campo magnético na horizontal, $H_x$, é apresentado na Equação 

\begin{eqnarray}
H_x &= & \frac{B_{\mathrm{r}}}{\pi \mu_{\mathrm{m}}}\left[\tan^{-1}\left(\frac{ab}{x(a^2+b^2+x^2)^{1/2}}\right) - \tan^{-1}\left(\frac{ab}{(x+L_0)(a^2+b^2+(x+L_0)^2)^{1/2}}\right)\right]
\end{eqnarray}

\paragraph{} O valor $x$ é a distância perpendicular do centro de um pólo do ímã até um ponto. É desejado que se saiba o vetor campo aplicado - um campo não horizontal - quando o ímã estiver com inclinação $\theta$ em relação ao horizonte e seu centro estiver numa posição $(x_0, y_0)$ em relação à origem. Considerando isto, chegamos na seguinte formulação:

\begin{eqnarray}
d &=& (x-x_0)\cos\theta + (y-y_0)\sin\theta\\
\mathbf{H}\cdot \mathbf{i} & = & H_d \cos \theta\\
\mathbf{H}\cdot \mathbf{j} & = & H_d \sin \theta\\
\end{eqnarray}


\subsection{Evolução do campo magnético}

Masao \cite{masao1989} cita Shliomis (cheguei a achar o artigo, mas era muito esquisito...) e mostra uma equação constitutiva para a evolução do campo magnético \begin{eqnarray}
\frac{\partial \mathbf{M}^*}{\partial t^*} = -\frac{1}{\tau}[\mathbf{M}^* - \mathbf{M}_0^*]+\frac{1}{\zeta}[(\mathbf{M}^*\times \mathbf{H}^*)\times\mathbf{M}^*]+\mathbf{\Omega^*}\times \mathbf{M}^*,
\end{eqnarray} na qual $\mathbf{\Omega}^*=\frac{1}{2}\nabla\times \mathbf{v}^*$ (qual a referência para isso? acho que Yuri não me mostrou). Nesta equação: $\Omega^*$ é a velocidade angular macroscópica do fluido; $\tau$ é o tempo de relaxação rotacional de movimento Browniano;  Todos os termos tem astericos indicando que são unidades dimensionais.

\subsubsection{Adimensionalizar}
\begin{eqnarray}
\frac{\partial \mathbf{M} H_0}{\partial t(L/U)} &=& -\frac{H_0}{\tau}[\mathbf{M} - \mathbf{M}_0]+\frac{H_0^3}{\zeta}[(\mathbf{M}\times \mathbf{H})\times\mathbf{M}]+\frac{H_0 U}{2L}(\nabla\times \mathbf{v})\times \mathbf{M}\\
\frac{\partial \mathbf{M}}{\partial t} &=& \frac{L}{U H_0}\left(-\frac{H_0}{\tau}[\mathbf{M} - \mathbf{M}_0]+\frac{H_0^3}{\zeta}[(\mathbf{M}\times \mathbf{H})\times\mathbf{M}]+\frac{H_0 U}{2L}(\nabla\times \mathbf{v})\times \mathbf{M}\right)\\
&=&-\frac{L}{\tau U}[\mathbf{M} - \mathbf{M}_0]+\frac{L H_0^2}{U\zeta}[(\mathbf{M}\times \mathbf{H})\times\mathbf{M}]+\frac{1}{2}(\nabla\times \mathbf{v})\times \mathbf{M}
\end{eqnarray}

\subsubsection{Discretização}
\paragraph{} Primeiro, note que \begin{eqnarray}
	\mathbf{u}\times \mathbf{v} & = & \left| \begin{array}{ccc}
 	\mathbf{i} & \mathbf{j} & \mathbf{k}\\
 	u_1 & u_2 & u_3\\
 	v_1 & v_2 & v_3\\
 \end{array}
 \right|.
\end{eqnarray}
\paragraph{} No caso $\mathbf{M}=(M_x, M_y, 0)$ e $\mathbf{H}=(H_x, H_y, 0)$, daí:

\begin{eqnarray}
	\mathbf{M}\times \mathbf{H} & = & \left| \begin{array}{ccc}
 	\mathbf{i} & \mathbf{j} & \mathbf{k}\\
 	M_x & M_y & 0\\
 	H_x & H_y & 0\\
 \end{array}
 \right| = (M_x H_y - M_y H_x)\mathbf{k}
\end{eqnarray}
assim
\begin{eqnarray}
(\mathbf{M}\times \mathbf{H})\times \mathbf{M} & = & \left| \begin{array}{ccc}
 	\mathbf{i} & \mathbf{j} & \mathbf{k}\\
 	0 & 0 & (M_x H_y - M_y H_x)\\
 	M_x & M_y & 0\\
 \end{array}
 \right|\\ 
 &=& -M_y(M_x H_y - M_y H_x)\mathbf{i}+M_x(M_x H_y - M_y H_x)\mathbf{j}
\end{eqnarray}

\paragraph{} O rotacional de $\mathbf{v} = (u, v, 0)$ é dado por \begin{eqnarray}
	\nabla\times \mathbf{v} & = & \left|\begin{array}{ccc}
\mathbf{i} & \mathbf{j} & \mathbf{k}\\
\frac{\partial}{\partial x} & \frac{\partial}{\partial y} & \frac{\partial}{\partial z}\\
u & v & 0\\
\end{array}
\right| = \left(\frac{\partial v}{\partial x} - \frac{\partial u}{\partial y}\right)\mathbf{k}
\end{eqnarray} e daí \begin{eqnarray}
(\nabla\times \mathbf{v})\times \mathbf{M} & = & \left|\begin{array}{ccc}
\mathbf{i} & \mathbf{j} & \mathbf{k} \\
0 & 0 & 	\left(\frac{\partial v}{\partial x} - \frac{\partial u}{\partial y}\right)\\
M_x & M_y & 0\\
\end{array}
\right| \\
& =& -M_y\left(\frac{\partial v}{\partial x} - \frac{\partial u}{\partial y}\right)\mathbf{i} + M_x\left(\frac{\partial v}{\partial x} - \frac{\partial u}{\partial y}\right)\mathbf{j}
\end{eqnarray}


\paragraph{} Pode-se definir as constantes \begin{eqnarray}
c_1 & = & \frac{L}{\tau U}\\
c_2 & = & \frac{LH_0^2}{U\zeta}
\end{eqnarray}

\paragraph{} Desta maneira:
\begin{eqnarray}
\frac{\partial M_x}{\partial t} & = & -c_1[M_x - M_{x_0}]-c_2 M_y(M_x H_y - M_y H_x)-\frac{1}{2}M_y\left(\frac{\partial v}{\partial x} - \frac{\partial u}{\partial y}\right)\\
\frac{\partial M_y}{\partial t} & = & -c_1[M_y - M_{y_0}]+c_2 M_x(M_x H_y - M_y H_x)+\frac{1}{2}M_x\left(\frac{\partial v}{\partial x} - \frac{\partial u}{\partial y}\right)
\end{eqnarray}

\subsection{Relação entre Magnetismo e Força}

\paragraph{} A relação entre força e o campo magnético é dada pelas equações abaixo:

\begin{align}
\mathbf{f} & = \mu_0\mathbf{M}\cdot \nabla \mathbf{H}\\ \eqname{Caso dimensional}\\
\mathbf{f} & = \mathrm{C}_{\mathrm{pm}}\,\mathbf{M}\cdot \nabla \mathbf{H}\\ \eqname{Caso adimensional}\\
\mathrm{C}_{\mathrm{pm}} & = \frac{\mu_0 H_0^2}{\rho u^2} \\ \eqname{Coeficiente de permeabilidade magnética}
\end{align}

\paragraph{} Desta forma, cada componente da força toma a seguinte forma:

\begin{eqnarray}
f_x & = & \mathrm{C}_{\mathrm{pm}}\left[M_x\cdot \frac{\partial H_x}{\partial x}+M_y\cdot \frac{\partial H_x}{\partial y}\right]\\
f_y & = & \mathrm{C}_{\mathrm{pm}}\left[M_x\cdot \frac{\partial H_y}{\partial x}+M_y\cdot \frac{\partial H_y}{\partial y}\right]
\end{eqnarray}

\paragraph{} A pesquisa se centra no $\mathrm{M}$, que depende do campo magnético, da velocidade do escoamento e do material. Na malha escalonada, $\mathbf{H}$ e $\mathbf{M}$ serão localizados da mesma forma que a velocidade $\mathbf{v}$.

\subsection{Condições de Contorno}
\paragraph{} A componente normal de $\mathbf{B}$ deve ser contínua através da interface entre o meio de fora (ar) e o de dentro (fluido magnético). Isto é

\begin{equation}
\left(\mathbf{B}_{\mathrm{f}} - \mathbf{B}_{\mathrm{d}}\right)\cdot \mathbf{n} = 0
\end{equation}

\paragraph{} Assim

\begin{equation}
\mu_0 (H_{\mathrm{f}}^{\mathrm{n}} + M_{\mathrm{f}}^{\mathrm{n}}) =  \mu_0(H_{\mathrm{d}}^{\mathrm{n}}+M_{\mathrm{d}}^{\mathrm{n}})
\end{equation}

\paragraph{} O meio de fora é o ar, logo, a magnetização $M_{\mathrm{f}}$ é zero. Sabe-se que $H = -\nabla\phi$, assim

\begin{eqnarray}
\nabla\phi_{\mathrm{d}}^{\mathrm{n}} & = & -H_{\mathrm{f}}^{\mathrm{n}} + M_{\mathrm{d}}^{\mathrm{n}}
\end{eqnarray}

\paragraph{} Desta forma

\begin{eqnarray}
\left.\frac{\partial \phi}{\partial x}\right|_{\mathrm{left}}\;\;&\approx&- H^{x}_{2,j} + M^{x}_{2,j}\\
\left.\frac{\partial \phi}{\partial x}\right|_{\mathrm{right}}&\approx&- H^{x}_{n,j} + M^{x}_{n,j}\\
\left.\frac{\partial \phi}{\partial y}\right|_{\mathrm{lower}}&\approx&- H^{y}_{i,2} + M^{y}_{i,2}\\
\left.\frac{\partial \phi}{\partial y}\right|_{\mathrm{upper}}&\approx&- H^{y}_{i,n} + M^{y}_{i,n}
\end{eqnarray}


\end{document}
