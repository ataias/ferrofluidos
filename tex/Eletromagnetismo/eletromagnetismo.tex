\documentclass[eletromagnetismo.tex]{subfiles}
\begin{document}
\section{Eletromagnetismo}

\paragraph{} Esta é a hora na qual as equações do eletromagnetismo serão introduzidas neste projeto. Até a última seção, a força utilizada era nula. A partir deste momento, iniciar-se-á um estudo matemático do comportamento de um fluido magnético na presença de um campo magnético. Iniciaremos com algoritmos básicos e não tão precisos e após isso iremos melhorar nosso ferramental. 

\subsection{Equações Básicas}

\paragraph{} Primeiramente, temos a equação do campo magnético e de seu divergente. O fato do divergente ser 0 indica que o campo é solenoidal, isto é, todas as linhas de campo magnético que saem, tem de voltar à superfície.

\begin{eqnarray}
\mathbf{B} &=& \mu_0(\mathbf{M}+\mathbf{H})\\
\nabla\cdot \mathbf{B} & = & 0
\end{eqnarray}

\paragraph{} A consequência das equações acima é que 

\begin{eqnarray}
\nabla\cdot \mathbf{M} = -\nabla\cdot\mathbf{H}	\label{mh1}
\end{eqnarray}


No caso de nosso estudo, temos um caso magnetostático, o que implica que:

\begin{eqnarray}
\nabla\times \mathbf{H} & = & \mathbf{0}
\end{eqnarray}

\paragraph{} Lembre-se que $\mathbf{H}$ é o fluxo elétrico. Como este campo é irrotacional, temos que $\mathbf{H}$ é um campo potencial:

\begin{equation}
\mathbf{H} = -\nabla\phi \label{poth}	
\end{equation}

\paragraph{} A partir das Equações \ref{mh1} e \ref{poth}, tem-se:

\begin{align}
\nabla^2\phi &= \nabla\cdot \mathbf{M}\label{eqcampomag}\\ \eqname{Equação do Campo Magnético}
\end{align}

\subsection{Relação entre Magnetismo e Força}

\paragraph{} A relação entre força e o campo magnético é dada pelas equações abaixo:

\begin{align}
\mathbf{f} & = \mu_0\mathbf{M}\cdot \nabla \mathbf{H}\\ \eqname{Caso dimensional}\\
\mathbf{f} & = \mathrm{Cpm}\,\mathbf{M}\cdot \nabla \mathbf{H}\\ \eqname{Caso adimensional}\\
\mathrm{Cpm} & = \frac{\mu_0 H_0^2}{\rho u^2} \\ \eqname{Coeficiente de permeabilidade magnética}
\end{align}

\paragraph{} Desta forma, cada componente da força toma a seguinte forma:

\begin{eqnarray}
f_x & = & \mathrm{Cpm}\left[M_x\cdot \frac{\partial H_x}{\partial x}+M_y\cdot \frac{\partial H_x}{\partial y}\right]\\
f_y & = & \mathrm{Cpm}\left[M_x\cdot \frac{\partial H_y}{\partial x}+M_y\cdot \frac{\partial H_y}{\partial y}\right]
\end{eqnarray}

\paragraph{} A pesquisa se centra no $\mathrm{M}$, que depende do campo magnético, da velocidade do escoamento e do material. Na malha escalonada, $\mathbf{H}$ e $\mathbf{M}$ serão localizados da mesma forma que a velocidade $\mathbf{v}$.

\subsection{Campo Aplicado}
\paragraph{} Para definir o campo aplicado $\mathbf{H}$, nos basearemos no trabalho de McCaig e Clegg \cite{mccaig}. O livro citado apresenta uma fórmula que condiz com a realidade muito bem. O campo magnético na horizontal, $H_x$, é apresentado na Equação 

\begin{eqnarray}
H_x &= & \frac{B_{\mathrm{r}}}{\pi \mu_{\mathrm{m}}}\left[\tan^{-1}\left(\frac{ab}{x(a^2+b^2+x^2)^{1/2}}\right) - \tan^{-1}\left(\frac{ab}{(x+L_0)(a^2+b^2+(x+L_0)^2)^{1/2}}\right)\right]
\end{eqnarray}

\paragraph{} O valor $x$ é a distância perpendicular do centro de um pólo do ímã até um ponto. É desejado que se saiba o vetor campo aplicado - um campo não horizontal - quando o ímã estiver com inclinação $\theta$ em relação ao horizonte e seu centro estiver numa posição $(x_0, y_0)$ em relação à origem. Considerando isto, chegamos na seguinte formulação:

\begin{eqnarray}
d &=& (x-x_0)\cos\theta + (y-y_0)\sin\theta\\
\mathbf{H}\cdot \mathbf{i} & = & H_d \cos \theta\\
\mathbf{H}\cdot \mathbf{j} & = & H_d \sin \theta\\
\end{eqnarray}

\subsection{Condições de Contorno}
\paragraph{} A componente normal de $\mathbf{B}$ deve ser contínua através da interface entre o meio de fora (ar) e o de dentro (fluido magnético). Isto é

\begin{equation}
\left(\mathbf{B}_{\mathrm{f}} - \mathbf{B}_{\mathrm{d}}\right)\cdot \mathbf{n} = 0
\end{equation}

\paragraph{} Assim

\begin{equation}
\mu_0 (H_{\mathrm{f}}^{\mathrm{n}} + M_{\mathrm{f}}^{\mathrm{n}}) =  \mu_0(H_{\mathrm{d}}^{\mathrm{n}}+M_{\mathrm{d}}^{\mathrm{n}})
\end{equation}

\paragraph{} O meio de fora é o ar, logo, a magnetização $M_{\mathrm{f}}$ é zero. Sabe-se que $H = -\nabla\phi$, assim

\begin{eqnarray}
\nabla\phi_{\mathrm{d}}^{\mathrm{n}} & = & -H_{\mathrm{f}}^{\mathrm{n}} + M_{\mathrm{d}}^{\mathrm{n}}
\end{eqnarray}

\paragraph{} Desta forma

\begin{eqnarray}
\left.\frac{\partial \phi}{\partial x}\right|_{\mathrm{left}}\;\;&\approx&- H^{x}_{2,j} + M^{x}_{2,j}\\
\left.\frac{\partial \phi}{\partial x}\right|_{\mathrm{right}}&\approx&- H^{x}_{n,j} + M^{x}_{n,j}\\
\left.\frac{\partial \phi}{\partial y}\right|_{\mathrm{lower}}&\approx&- H^{y}_{i,2} + M^{y}_{i,2}\\
\left.\frac{\partial \phi}{\partial y}\right|_{\mathrm{upper}}&\approx&- H^{y}_{i,n} + M^{y}_{i,n}
\end{eqnarray}

\subsection{Procedimento}
\paragraph{} O procedimento para se resolver a Equação de Navier Stokes com o campo magnético, nas condições simples apresentadas, é:

\begin{enumerate}
	\item Resolver equação do campo magnético, se for passo inicial: $\mathbf{M} = \mathbf{0}$
	\item Calcular $\mathbf{M} = \chi \mathbf{H}$ (lei simples)
	\item Calcular força $\mathbf{f} = \mathrm{Cpm}\,\mathbf{M}\cdot \nabla \mathbf{H}$
	\item Resolver hidrodinâmica
	\item Avançar no tempo e voltar ao passo 1
\end{enumerate}


\end{document}
