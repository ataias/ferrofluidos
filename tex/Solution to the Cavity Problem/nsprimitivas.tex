\documentclass[poisson.tex]{subfiles}
\begin{document}
\section{Resolução Numérica}
\subsection{Desconsiderando a pressão}
\paragraph{} A velocidade $\textbf{v}^*$ não leva em conta a pressão. Primeiramente, discretizamos a Equação  \ref{equacaovestrela} por meio de diferenças finitas. A parte da direita de cada equação é avaliada no tempo passado, enquanto a parte da esquerda é o resultado no tempo estrela.

\begin{eqnarray}
\textbf{v}_{ij}^s&=&(\SumFour{u})\textbf{i} \nonumber \\&+& (\SumFour{v})\textbf{j}\\
\textbf{v}_{ij}^t&=&0.25(u_{ij}^n+u_{i+1j}^n+u_{i+1j-1}^n+u_{ij-1}^n)\textbf{i} \nonumber \\ &+& 0.25(v_{ij}^n+v_{i-1j}^n+v_{i-1j+1}^n+v_{ij+1}^n)\textbf{j}
\end{eqnarray}
%------------------- u* -------------------
\begin{eqnarray}
u_{ij}^{*}&=&u_{ij}^n+\frac{\Delta t}{\rho}\left[\mu\left(\frac{u_{ij}^s-4u_{ij}^n}{\Delta
x^2}\right)+Fx_{ij}^n\right] + \nonumber \\
&-&\left[u_{ij}^n(u_{i+1j}^n-u_{i-1j}^n)+v_{ij}^t(u_{ij+1}^n-u_{ij-1}^n)\right]\frac{\Delta t}{2\Delta x}
\end{eqnarray}
%------------------- v* ------------------
\begin{eqnarray}
v_{ij}^{*}&=&v_{ij}+\frac{\Delta t}{\rho}\left[\mu\left(\frac{v_{ij}^s-4v_{ij}^n}{\Delta
x^2}\right)+Fy_{ij}^n\right] \nonumber \\
&-&\left[u_{ij}^t (v_{i+1j}^n-v_{i-1j}^n)+v_{ij}^n(v_{ij+1}^n-v_{ij-1}^n)\right]\frac{\Delta t}{2\Delta x}
\end{eqnarray}

\paragraph{} Note que as equações acima são só válidas para pontos internos à malha escalonada. Os pontos de fronteira  não sofrem alteração nenhum nesta etapa. É importante atender às condições de não deslizamento e impenetrabilidade em todo momento. 

%------------------------Obter a pressão!!!! --------------------
\subsection{Obter a pressão}
\paragraph{} Como já foi apresentado na seção de introdução, a segunda etapa é obter a pressão, por meio da Equação \ref{pressaoEq}.
\begin{equation}
\nabla^2 p = \frac{\rho}{\Delta t} \nabla\cdot \textbf{v}^*=\frac{\rho}{\Delta t} 
\left( \frac{\partial u^*}{\partial x}+\frac{\partial v^*}{\partial y} \right) \label{pressaoEq}
\end{equation}
\paragraph{} Para cada ponto $ij$ da malha, deve-se calcular a não-homogeneidade apresentada na direita da Equação \ref{pressaoEq}:
\begin{equation}
DIV_{ij}=\frac{\rho}{\Delta t}\left( \frac{u_{i+1j}^*-u_{ij}^*}{\Delta x}+\frac{v_{ij+1}^*-v_{ij}^*}{\Delta x}\right)
\end{equation}
\paragraph{} A equação \ref{pressaoEq} é uma equação de Poisson. Para pontos internos à malha, tem-se:

\begin{equation}
p_{ij}=\frac{1}{4}[(p_{i+1j}+p_{i-1j}+p_{ij+1}+p_{ij-1})-\Delta x^2 DIV_{ij}]
\end{equation}

\paragraph{} O que acontece com as condições de fronteira? Ainda não sabemos quais são elas, é necessário que sejam obtidas antes de se resolver a equação de Poisson. Para obter tais condições, considere novamente a equação de Navier Stokes:

\[\rho\left( \frac{\partial \textbf{v}}{\partial t}+\textbf{v}\cdot\nabla\textbf{v}\right)=-\nabla p+\mu\nabla^2\textbf{v}+\textbf{f}\]
\[\nabla\cdot\textbf{v}=0\]
\paragraph{} Destrinchando a equação para duas coordenadas, tem-se:
\begin{eqnarray}
\frac{\partial u}{\partial t}+u\frac{\partial u}{\partial x}+v\frac{\partial
u}{\partial y}=-\frac{1}{\rho}\frac{\partial p}{\partial
x}+\nu\left(\frac{\partial^2 u}{\partial^2 x}+\frac{\partial^2 u}{\partial^2
y}\right)+f_x\\
\frac{\partial v}{\partial t}+u\frac{\partial v}{\partial x}+v\frac{\partial
v}{\partial y}=-\frac{1}{\rho}\frac{\partial p}{\partial
y}+\nu\left(\frac{\partial^2 v}{\partial^2 x}+\frac{\partial^2 v}{\partial^2
y}\right)+f_y
\end{eqnarray}
\paragraph{} Se fizermos uma análise dos pontos na parede da esquerda ou
direita, nossa fronteira, e aplicarmos as condições de impenetrabilidade e não-deslizamento, que
dizem que $u=0$ e $v=0$ na parede, temos a simplificação da equação, pois vários
termos se tornam zero. As Equações \ref{paredesVerticais} e \ref{paredesHorizontais} são as condições de fronteira, tipo Neumann, para as paredes verticais e horizontais, respectivamente.

\begin{eqnarray}
\frac{\partial p}{\partial x}\Bigg|_{\textrm{parede}}&=&\rho\nu\frac{\partial^2
u}{\partial x^2}\Bigg|_{\textrm{parede}}+\rho f_x\label{paredesVerticais}\\
\frac{\partial p}{\partial y}\Bigg|_{\textrm{parede}}&=&\rho\nu\frac{\partial^2
v}{\partial y^2}\Bigg|_{\textrm{parede}}+\rho f_y\label{paredesHorizontais}
\end{eqnarray}

\paragraph{} O próximo passo é reescrever as Equações \ref{paredesVerticais}  e \ref{paredesHorizontais} com diferenças finitas. Para isso, foram utilizadas diferenças finitas unilaterais, para as velocidades, e centrais, para a pressão. A discretização tem ordem 2. As equações, em
sequência, para as fronteiras da esquerda, direita, baixo e topo da matriz estão nas Equações \ref{pressure_01} até \ref{pressure_04}. Todas as discretizações foram feitas com malha escalonada, lembre-se!

\begin{eqnarray}
\frac{p_{ij}-p_{i-1j}}{\Delta x}&=&\frac{\mu}{\Delta x^2}(2u_{ij}-5u_{i+1j}+4u_{i+2j}-u_{i+3j})+\rho f_{x,ij} \label{pressure_01}\\
\frac{p_{ij}-p_{i-1j}}{\Delta x}&=&\frac{\mu}{\Delta x^2}(2u_{ij}-5u_{i-1j}+4u_{i-2j}-u_{i-3j})+\rho f_{x,ij}\label{pressure_02}\\
\frac{p_{ij}-p_{ij-1}}{\Delta y}&=&\frac{\mu}{\Delta y^2}(2v_{ij}-5v_{ij+1}+4v_{ij+2}-v_{ij+3})+\rho f_{y,ij} \label{pressure_03}\\
\frac{p_{ij}-p_{ij-1}}{\Delta y}&=&\frac{\mu}{\Delta y^2}(2v_{ij}-5v_{ij-1}+4v_{ij-2}-v_{ij-3})+\rho f_{y,ij} \label{pressure_04}
\end{eqnarray}

\paragraph{} Reescrevendo-se as equações \ref{pressure_01} até \ref{pressure_04}, e considerando $n$ a ordem da matriz, temos:
\begin{eqnarray}
i=1 \,|\, p_{i-1j}&=&p_{ij}-\frac{\mu}{\Delta x}\left(2u_{ij}-5u_{i+1j}+4u_{i+2j}-u_{i+3j}\right)-\rho\Delta x f_{x,ij}\\
i=n \,|\, p_{ij}&=&p_{i-1j}+\frac{\mu}{\Delta x}\left(2u_{ij}-5u_{i-1j}+4u_{i-2j}-u_{i-3j}\right)+\rho\Delta x f_{x,ij}\\
j=1 \,|\, p_{ij-1}&=&p_{ij}-\frac{\mu}{\Delta x}\left(2v_{ij}-5v_{ij+1}+4v_{ij+2}-v_{ij+3}\right)-\rho\Delta x f_{y,ij}\\
j=n \,|\, p_{ij}&=&p_{ij-1}+\frac{\mu}{\Delta x}\left(2v_{ij}-5v_{ij-1}+4v_{ij-2}-v_{ij-3}\right)+\rho\Delta x f_{y,ij}
\end{eqnarray}
\paragraph{} Dependendendo se a linguagem que você usa inicia índices de 0 ou 1, você pode precisar mudar um pouco os índices $i$ e $j$ acima, mas as equação permanecem as mesmas. 
\subsection{Resolver para o passo $t+\Delta t$}
\paragraph{} Finalmente, tem-se: 

\begin{equation}
\frac{\textbf{v}^{n+1}-\textbf{v}^*}{\Delta t}=-\frac{1}{\rho}\nabla p
\end{equation}
\paragraph{} Expandindo para as equações escalares, tem-se:
\begin{eqnarray}
u^{n+1}=u^*-\frac{\Delta t}{\rho}\frac{\partial p}{\partial x}\\
v^{n+1}=v^*-\frac{\Delta t}{\rho}\frac{\partial p}{\partial y}
\end{eqnarray}
\paragraph{} Basta discretizar os termos das derivadas parciais agora:

\begin{eqnarray}
p_{x,ij}=\frac{p_{ij}-p_{i-1j}}{\Delta x}\\
p_{y,ij}=\frac{p_{ij}-p_{ij-1}}{\Delta x}
\end{eqnarray}

\paragraph{} E finalmente tem-se a solução para o tempo $n+1$ a partir do tempo passado $n$:

\begin{eqnarray}
\NextTimeStep{u}{x}\\
\NextTimeStep{v}{y}
\end{eqnarray}
\paragraph{} As equações acima são válidas para pontos internos. Para as fronteiras, comece considerando a parede da esquerda da cavidade, onde a velocidade normal é 0. Como nossa malha é escalonada, este ponto não está salvo na matriz e uma média deve ser feita de modo a determinar o ponto da fronteira da malha. Neste caso, teria-se:

\begin{eqnarray*}
\frac{v_{i-\frac{1}{2}j}+v_{i+\frac{1}{2}j}}{2}=0
\end{eqnarray*}

o que leva a:
\begin{eqnarray}
v_{i-\frac{1}{2}j} &=& -v_{i+\frac{1}{2}j}
\end{eqnarray}

De maneira similar, para as paredes de baixo e da direita, tem-se:

\begin{eqnarray}
u_{ij-\frac{1}{2}}&=&-u_{ij+\frac{1}{2}}\\
v_{i+\frac{1}{2}j} &=& -v_{i-\frac{1}{2}j}
\end{eqnarray}

Para a parte de cima, onde a velocidade é diferente de zero, tem-se:

\begin{eqnarray*}
\frac{u_{ij+\frac{1}{2}}+u_{ij-\frac{1}{2}}}{2}=u_{Bi}
\end{eqnarray*}

$u_{Bi}$ é a condição de fronteira para o ponto na coluna $i$.


Retirando os índices fracionários:

\begin{eqnarray}
v_{i-1j} &=& -v_{ij},\,\, \textrm{esquerda}\\
u_{ij-1}&=&-u_{ij},\,\, \textrm{embaixo} \\
v_{ij} &=& -v_{i-1j},\,\, \textrm{direita}\\
u_{ij} &=& 2u_{Bi}-u_{ij-1},\,\, \textrm{em cima}
\end{eqnarray}

fixando o índice de cada lado:

\begin{eqnarray}
v_{0j} &=& -v_{1j},\,\, \textrm{esquerda}\\
u_{i0}&=&-u_{i1},\,\, \textrm{embaixo} \\
v_{n-1j} &=& -v_{n-2j},\,\, \textrm{direita}\\
u_{in-1} &=& 2u_{Bi}-u_{in-2},\,\, \textrm{em cima}
\end{eqnarray}

\end{document}
