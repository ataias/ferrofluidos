\documentclass[fisica.tex]{subfiles}
\begin{document}
\section{Física}
\paragraph{} Antes de iniciar-se a resolução numérica da equação de Navier Stokes, seja qual for o método escolhido, é necessário ter em mente que certas condições físicas devem ser satisfeitas de modo a se ter um resultado que convirja numericamente. As restrições irão determinar quais os mínimos/máximos de $\Delta t$ e $\Delta x$. Nesta seção veremos as condições de difusão estável, advecção (ou transporte) estável e camada limite hidrodinâmica.

\paragraph{} Além de restrições físicas, iremos também introduzir o número de Reynolds: $\mathit{Re}$. Este é um número adimensional no qual se escolhe uma velocidade e tamanho característicos para o problema em questão. O número de Reynolds é dado por:

\begin{eqnarray}
\mathit{Re}=\frac{\rho L U}{\mu}
\end{eqnarray}

\paragraph{} Neste trabalho, tem-se um quadrado de lado 1. Seja então $L=1$ o tamanho característico e $U=1$ a velocidade característica (note que esse $U$ é a velocidade que multiplica a condição de fronteira $u(x,1)=U\sin{\pi x}$). Ainda, escolha $\rho = 1$. Desta forma:

\begin{eqnarray}
\mathit{Re}=\frac{1}{\mu}
\end{eqnarray}
 
\paragraph{} As equações de Navier Stokes são discretizadas adiante com $\rho$ e $\mu$. Utilizaremos o número de Reynolds por meio da técnica mostrada aqui e variar-se-á somente $\mu$ para influenciar as características do escoamento.

\subsection{Difusão Estável}
\begin{eqnarray}
\Delta t < \frac{1}{4}\mathit{Re}\Delta x^2
\end{eqnarray}

\subsection{Advecção Estável}
\begin{eqnarray}
\Delta t < \frac{\Delta x}{U}
\end{eqnarray}

\subsection{Camada Limite Hidrodinâmica}
\paragraph{} O que é essa camada mesmo?
\begin{eqnarray*}
\mathit{Re}_{\Delta x} &=& \frac{\rho U \Delta x}{\mu} < 1\\
\mathit{Re}_{\Delta x} &=& \frac{\rho U \Delta x}{\mu} \frac{L}{L}\\
&=& \frac{\rho U L}{\mu} \frac{\Delta x}{L}=\mathit{Re}\cdot \lambda < 1
\end{eqnarray*}

\paragraph{} Assim

\begin{eqnarray}
\lambda < \frac{1}{\mathit{Re}}
\end{eqnarray}


\end{document}
